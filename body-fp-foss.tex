%% Talk on Functional Programming and FOSS for Linux Users of Victoria, 2016-11-02.
%% http://luv.asn.au/
%% Les Kitchen <luv-fp-foss@po.ljk.id.au>
%%

%% Body of talk, to be input.
\mode<presentation>
{
  \usetheme{default}
}

\usepackage{graphicx}
\usepackage[english]{babel}
\usepackage{listings}

%
\newcommand{\hide}[1]{}% Make things disappear
%
% Comments to be supressed for final version
% Conflicted with \comment in html.sty, so made \Comment
\newcommand{\Comment}[1]{{{$\Longrightarrow$} #1 {$\Longleftarrow$}}}
%
\newcommand{\signoff}[1]% Sign-off lines, with #1 date
{%
\begin{flushright}
Les Kitchen, #1
\end{flushright}%
}
%
\newcommand{\bc}{\begin{center}}
\newcommand{\ec}{\end{center}}
% Causes conflict with picinpar package
%%%\newcommand{\br}{\begin{flushright}}
%%%\newcommand{\er}{\end{flushright}}
\newcommand{\bi}{\begin{itemize}}
\newcommand{\ei}{\end{itemize}}
\newcommand{\be}{\begin{enumerate}}
\newcommand{\ee}{\end{enumerate}}
\newcommand{\bd}{\begin{description}}
\newcommand{\ed}{\end{description}}
\newcommand{\bv}{\begin{verse}}
\newcommand{\ev}{\end{verse}}
%
\newcommand{\lb}{\linebreak[0]}
%
\def\monthname{\ifcase\month\or
	January\or February\or March\or April\or May\or June\or
	July\or August\or September\or October\or November\or December\fi}
\renewcommand{\today}{\number\day\ \monthname\ \number\year}
\newcommand{\dated}[1]% Force a particular date
{\renewcommand{\today}{#1}}
%
\newcommand{\pg}[1]{\paragraph*{#1}}
\newcommand{\T}[1]{\texttt{#1}}
\newcommand{\TT}[1]{\begin{verse}\texttt{#1}\end{verse}}

%
\newcommand{\picframe}[2][0.8]{\frame<presentation>[plain]{\bc\includegraphics[width=#1\textwidth]{#2}\ec}}
\newcommand{\figframe}[4][0.8]{\frame[plain]{\begin{figure}\begin{center}\includegraphics[width=#1\linewidth]{#2}\end{center}\only<article>{\caption{#4}}\label{#3}\end{figure}}}
\newcommand{\txtframe}[1]{\frame<presentation>[plain]{\bc#1\ec}}
\newenvironment{annotation}%
{\begin{onlyenv}<article>%
  \par\noindent\hrulefill\quad\raisebox{-0.5ex}{Notes}\quad\hrulefill\smallskip\par\noindent}
{\smallskip\par\noindent\hrule\end{onlyenv}}


\title[FP\&FOSS]
{Functional Programming and FOSS}

%%%\subtitle{Something}

\author{Les Kitchen \\
  {\small Department of Computing and Information Systems} \\
  University of Melbourne \\
  Linux Users of Victoria \\
  \texttt{luv-fp-foss@po.ljk.id.au}}

\date{2~November, 2016}

\begin{document}

\begin{frame}<presentation>
  \titlepage
\end{frame}

\only<article>{\maketitle}

\only<article>{
\begin{center}
\em Here is the material used in my talk,
  along with additional commentary.
\end{center}}

\begin{frame}<presentation>
  \frametitle{Agenda}
  \tableofcontents
  % You might wish to add the option [pausesections]
\end{frame}

%\only<article>{\tableofcontents}

\begin{annotation}
  An introduction to functional programming, historical,
  philosophical, and practical, linking up with FOSS, Free and
  Open-Source Software.
\end{annotation}

\section{The Four Programming Paradigms}

\txtframe{\Huge The Four Paradigms}

\begin{frame}
%%%\frametitle{The Four Paradigms}
\bc
\begin{tabular}{@{}c|c@{}}
\begin{tabular}[t]{@{}c@{}}
{\huge Imperative} \\
\uncover<2->{Fortran, 1953{\slash}1957} \\
\uncover<2->{Algol, 1958{\slash}1960}
\end{tabular}
&
\begin{tabular}[t]{@{}c@{}}
{\huge Object-Oriented} \\
\uncover<2->{Simula, 1967}
\end{tabular}
\\
\hline
\\
\begin{tabular}[t]{@{}c@{}}
{\huge Logic} \\
\uncover<2->{Planner, 1969} \\
\uncover<2->{Prolog, 1972}
\end{tabular}
&
\begin{tabular}[t]{@{}c@{}}
{\huge Functional} \\
\uncover<2->{Lisp, 1958{\slash}1962}
\end{tabular}
\end{tabular}
\ec
\begin{annotation}
To see how functional programming (FP) fits into the overall
scheme of programming languages, I'll introduce the concept of
\emph{The Four Programming Paradigms}, which you might already
have encountered.  Under this concept, there are four main ways
of programming:
\bd
\item[Imperative Programming] A program is thought of as
a sequence of instructions to the computer to do things,
typically to change (mutate) the data stored in variables by
assignment (or to do input and output).
Hence, \emph{Imperative}.  Typically, the data is organized into
data structures, like records and arrays, and the instructions
are organized into what are variously called subroutines,
functions, or procedures (with some mechanism for passing
arguments).  Hence, this is sometimes also referred to
as \emph{Procedural Programming}.  There are also control
structures, like conditionals and loops, to execute instructions
selectively or repetitively (\emph{iteration}).

\item[Object-Oriented Programming] A running program is thought of as a
collection of \emph{objects}, which bundle data and procedures
(usually called \emph{methods}).  The program runs by the
objects sending messages to each other to invoke methods.  Aside
from this object-oriented organizing principle, an
object-oriented program works much like an imperative program,
by mutating the state of objects, with conditionals and loops.

\item[Logic Programming]  A program is thought of as collection
of \emph{facts} and inference \emph{rules} in some formal logic
notation.  The program runs essentially by (multi-step)
inferences from the facts using the rules, to produce new
(useful) facts.

\item[Functional Programming]  A program is thought of as
a collection of \emph{functions} (sometimes confusingly
called \emph{procedures}).  The program runs by functions
invoking functions, often recursively.  The emphasis is on the
values computed by the functions, not on any side-effects of
mutating variables.  In some functional programming languages
such side-effects are expressly forbidden (these are
called \emph{pure}); in others (\emph{impure}), side-effects are
allowed, but are generally discouraged.  Characteristic of
functional programming languages is that functions are ``first
class'', that is, they can be treated much as data in imperative
languages: they can be passed as arguments to functions,
computed and returned as results from functions, and stored into
data-structures.  This leads to very powerful programming
techniques.  Also characteristic of functional programming
languages is the use of recursion instead of iteration.  Even
when they provide iterative constructs, they are usually just
syntactic sugar for some underlying recursive schema.
\ed
As you can see from the various programming languages mentioned
in the slide, these Four Paradigms go back to the early,
sometimes very early, days of modern computing.  The years
mentioned are only indicative.  A single year is typically the
year in which the language was first released, with a working
compiler or interpreter.  Where there are two years, like
1958{\slash}1962 for Lisp, the first year is when there was a
substantial proposal or design for the language, and the second
is the year of first release of a working compiler or
interpreter.

Obviously, there are no sharp boundaries between these
paradigms, and the languages that claim them.  It's largely a
matter of emphasis.  Even a low-level, notionally imperative
language like C provides some limited capabilities for doing
functional programming.  The first implementation of what we
would call logic programming was done as an implementation of
Planner embedded inside Lisp, an impure functional language
which also supports imperative constructs.  The pure functional
language Haskell provides constructs which permit programming in
an imperative style where appropriate (although these constructs
are ultimately defined in pure functional terms).  There's
interest in adapting ``advanced'' functional techniques like
monads to languages like Javascript.  These cross-over examples
can be multiplied almost endlessly.

Even though there is a lot of cross-over, The Four Paradigms
provide a useful frame of reference, and most programming
languages can be seen as fitting into one or other of these
paradigms (with varying strictness), or of providing
capabilities from the paradigms.

As mentioned, imperative and object-oriented programming
typically use iteration (loops), while functional and logic
programming typically use recursion.  While Algol'60 is
notionally an imperative language, I point out as something of
an historical anecdote that in the Algol'60 report (the defining
document for the language), iteration by the \texttt{while} loop
is actually defined in terms of recursion.  In a way, this is
not surprising, since one of the co-authors of the Algol'60
report was John McCarthy, who was also creator of the functional
language Lisp.
\end{annotation}
\end{frame}

%%%\txtframe{\Huge Declarative Programming}
\begin{annotation}
Functional programming and logic programming are often
collectively referred to as \emph{declarative programming},
since the emphasis is on declaring the functional or
logical relationships in the program, and not so much on the
step-by-step execution of the program through changing state.
\end{annotation}

\section{Functional Programming}

\txtframe{\Huge Functional Programming}

\begin{frame}
\frametitle{Functional Programming}
Programming with ``first class'' functions
\bi
\item functions as function arguments
\item functions as function results
\item functions as data
\item ``higher order'' functions
\item abstraction of data and control
\ei
\end{frame}

\begin{annotation}
Stuff.
\end{annotation}

\begin{frame}
\frametitle{FP variations}
\bi
\item pure versus impure
\item static versus dynamic typing
\item lazy versus eager evaluation
\item lexical versus dynamic binding
\ei
\end{frame}

\begin{frame}
\frametitle{FP ``extremes''}
\bi
\item Traditional Lisp
\bi
\item impure, dynamic typing, eager evaluation, dynamic binding
\ei
\item Haskell
\bi
\item pure, static typing, ``lazy'' evaluation, lexical binding
\ei
\ei
\end{frame}

\section{History of FP}

\begin{frame}
\frametitle{Sketch history of FP}
\framesubtitle{60 years in the making, or longer...}
\bi
\item Alonzo Church, $\lambda$-calculus, 1930s
\item John McCarthy, Lisp, 1950s
\item various Lisp-like languages
\bi
\item Scheme (1970s), Common Lisp (1980s), Clojure (2000s)
\ei
\item statically typed FP languages
\bi
\item ML, SML, Miranda, OCaml, Haskell
\ei
\item also: Erlang, Scala
\ei
\begin{annotation}
Stuff.
\end{annotation}
\end{frame}

\section{Why FP?}

\begin{frame}
\frametitle{Why FP?}
\bi
\item Better for big systems
\bi
\item type-safety
\item correctness
\item controlled interaction
\item expressive power
\ei
\item Better for parallelism, multi-core, cache
\ei
\end{frame}

\section{Simple examples}

\begin{frame}
\frametitle{mapfl in Haskell}
\lstinputlisting[language=Haskell]{Mapfl.hs}
\end{frame}

\begin{frame}
\frametitle{mapfl in Scheme}
\lstinputlisting{mapfl.scm}
\end{frame}

\section{Quick sample tour of FP Zoo}

\begin{frame}
\frametitle{Scala}
\bi
\item Developed by Martin Odersky from 2001
\item ``Multi-paradigm'', but strong support for FP
\item Java-like syntax
\item JVM
\ei
\end{frame}

\begin{frame}
\frametitle{Clojure}
\bi
\item Developed by Rich Hickey, 2007
\item Lisp-like syntax
\item JVM
\bi
\item Clojurescript $\longrightarrow$ Javascript
\ei
\item Strong emphasis on
\bi
\item immutability
\item concurrency
\item persistent data structures
\ei
\ei
\end{frame}

\begin{frame}
\frametitle{Haskell}
\bi
\item Started by committee, 1987{\slash}1990
\bi
\item {\small Simon Peyton-Jones, Phil Wadler, John Hughes, \ldots}
\ei
\item Pure --- ``referential transparency''
\item Strong statically typed
\item Non-strict (``lazy'') evaluation
\item GHC compiler: multi-target + Javascript
\item Hugs, Yhc: byte-code
\ei
\end{frame}

\begin{frame}
\frametitle{Worth mentioning}
\bi
\item Common Lisp
\item Scheme (Guile, Racket)
\item OCaml
\item Clean
\item Mercury
\item Erlang
\item F\#
\ei
\end{frame}

\begin{frame}
\frametitle{FP at work}
\bi
\item Haskell
\bi
\item Darcs distributed version-control system
\item Xmonad window manager
\item Facebook anti-spam framework
\item GHC
\ei
\item Clojure
\bi
\item Australia Post, Silverpond, Thoughtworks, Zendesk, Walmart
      Labs, eBay, Facebook
\ei
\item Scala
\bi
\item REA Group
\ei
\item Mercury
\bi
\item YesLogic
\ei
\ei
\end{frame}

\begin{frame}{FP community in Melbourne}
All on Meetup:
\bi
\item Melbourne Functional User Group, MFUG
\item Melbourne Haskell User Group, MHUG
\item clj-melb (Clojure)
\item Melbourne Scala User Group, MSUG
\ei
Recent Compose Melbourne FP event, videos available.
\bi
\item \url{http://www.composeconference.org/}
\item videos: search for ``compose melbourne'' on Youtube
\ei
\end{frame}

\section{FP and FOSS}

\begin{frame}
\frametitle{FP and FOSS}
\bi
\item Haskell
\bi
\item GHC --- BSD 3-clause
\item Hugs --- BSD
\item Haskell Platform --- BSD
\ei
\item Clojure --- Eclipse Public Licence
\item Scala --- BSD 3-clause
\item Scheme --- Guile, LGPL; Racket, LGPL
\item Mercury --- GPL{\slash}LGPL
\item many others
\ei
\end{frame}

\section{Haskell}

\begin{frame}
\frametitle{Characteristics of Haskell}
\bi
\item concise, clean notation
\item strong static typing
\item pattern matching
\item currying
\item lazy evaluation
\item ``no side-effects''
\item monads \& IO
\item compiled --- good performance achievable
\ei
\end{frame}

\begin{frame}
\frametitle{Haskell's type system}
\bi
\item strong static typing
\item type inference
\item polymorphism
\item abstract datatypes
\item typeclasses
\ei
\end{frame}

\begin{frame}
\frametitle{Lazy evaluation}
\bi
\item evaluation ``by need''
\item not unlike Unix pipelines
\item allows ``infinite'' data structures, \texttt{[1..]}
\item strict evaluation possible
\ei
\end{frame}

\begin{frame}
\frametitle{Control structures by lazy evaluation}
If-Then-Else as an ordinary function:
\lstinputlisting{IfThenElse.hs}
\end{frame}

\begin{frame}
\frametitle{Maybe monad and failure}
\lstinputlisting{MaybeStuff.hs}
\end{frame}

\begin{frame}
\frametitle{IO monad and IO actions}
\lstinputlisting{Hello.hs}
%%%\texttt{main} has a value of type \texttt{IO~a} which is a
%%%description of the intended input-output behavior of the
%%%program, lazily realised.
\end{frame}

\begin{frame}
\frametitle{Lambda expressions \& currying}
\lstinputlisting{NoLambdaExample.hs}
\lstinputlisting{WithLambdaExample.hs}
\lstinputlisting{MapLambdaExample.hs}
\end{frame}

\begin{frame}
\frametitle{Tail-call optimization}
Straight recursive version:
\lstinputlisting{RecFac.hs}
Tail-recursive version:
\lstinputlisting{TailFac.hs}
\end{frame}

\begin{frame}
\frametitle{Haskell downsides}
\bi
\item Lazy evaluation:
\bi
\item run-time overhead
\item hard to predict resource usage
\item (used to make debugging difficult)
\item solution: judicious use of strictness and compiler optimizations
\ei
\item Sometimes confusing error messages
\item Monads don't compose well
\ei
\end{frame}

\begin{frame}
\frametitle{Summary}
\bi
\item FP context and history
\item FP advantages and characteristics
\item FP FOSS implementations
\ei
Work in progress: \\
{\small \url{https://github.com/LJKitchen/ljk-luv-fp-foss}}
\medskip
{\scriptsize Produced using the {\LaTeX} Beamer package, along
  with other free-software programs.}
\end{frame}

\end{document}
