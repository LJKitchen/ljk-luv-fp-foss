%% Talk on Functional Programming and FOSS for Linux Users of Victoria, 2016-11-02.
%% http://luv.asn.au/
%% Les Kitchen <luv-fp-foss@po.ljk.id.au>
%%

%% Body of talk, to be input.
\mode<presentation>
{
  \usetheme{default}
}

\usepackage{graphicx}
\usepackage[english]{babel}

%
\newcommand{\hide}[1]{}% Make things disappear
%
% Comments to be supressed for final version
% Conflicted with \comment in html.sty, so made \Comment
\newcommand{\Comment}[1]{{{$\Longrightarrow$} #1 {$\Longleftarrow$}}}
%
\newcommand{\signoff}[1]% Sign-off lines, with #1 date
{%
\begin{flushright}
Les Kitchen, #1
\end{flushright}%
}
%
\newcommand{\bc}{\begin{center}}
\newcommand{\ec}{\end{center}}
% Causes conflict with picinpar package
%%%\newcommand{\br}{\begin{flushright}}
%%%\newcommand{\er}{\end{flushright}}
\newcommand{\bi}{\begin{itemize}}
\newcommand{\ei}{\end{itemize}}
\newcommand{\be}{\begin{enumerate}}
\newcommand{\ee}{\end{enumerate}}
\newcommand{\bd}{\begin{description}}
\newcommand{\ed}{\end{description}}
\newcommand{\bv}{\begin{verse}}
\newcommand{\ev}{\end{verse}}
%
\newcommand{\lb}{\linebreak[0]}
%
\def\monthname{\ifcase\month\or
	January\or February\or March\or April\or May\or June\or
	July\or August\or September\or October\or November\or December\fi}
\renewcommand{\today}{\number\day\ \monthname\ \number\year}
\newcommand{\dated}[1]% Force a particular date
{\renewcommand{\today}{#1}}
%
\newcommand{\pg}[1]{\paragraph*{#1}}
\newcommand{\T}[1]{\texttt{#1}}
\newcommand{\TT}[1]{\begin{verse}\texttt{#1}\end{verse}}

%
\newcommand{\picframe}[2][0.8]{\frame<presentation>[plain]{\bc\includegraphics[width=#1\textwidth]{#2}\ec}}
\newcommand{\figframe}[4][0.8]{\frame[plain]{\begin{figure}\begin{center}\includegraphics[width=#1\linewidth]{#2}\end{center}\only<article>{\caption{#4}}\label{#3}\end{figure}}}
\newcommand{\txtframe}[1]{\frame<presentation>[plain]{\bc#1\ec}}
\newenvironment{annotation}%
{\begin{onlyenv}<article>%
  \par\noindent\hrulefill\quad\raisebox{-0.5ex}{Notes}\quad\hrulefill\smallskip\par\noindent}
{\smallskip\par\noindent\hrule\end{onlyenv}}


\title[FP\&FOSS]
{Functional Programming and FOSS}

%%%\subtitle{Something}

\author{Les Kitchen \\
  Department of Computing and Information Systems \\
  University of Melbourne}

\date{2~November, 2016}

\begin{document}

\begin{frame}<presentation>
  \titlepage
\end{frame}

\only<article>{\maketitle}

\only<article>{
\begin{center}
\em Here is the material used in my talk,
  along with additional commentary.
\end{center}}

\begin{frame}<presentation>
  \frametitle{Agenda}
  \tableofcontents
  % You might wish to add the option [pausesections]
\end{frame}

%\only<article>{\tableofcontents}

\begin{annotation}
  An introduction to functional programming, historical,
  philosophical, and practical, linking up with FOSS, Free and
  Open-Source Software.
\end{annotation}

\section{What is Functional Programming?}

\txtframe{\Huge Functional Programming}

\begin{frame}
\frametitle{Functional Programming}
\bi
\item Programming with functions
\ei
\end{frame}

\begin{annotation}
Stuff.
\end{annotation}

\section{The Four Programming Paradigms}

\txtframe{\Huge The Four Paradigms}

\begin{frame}
%%%\frametitle{The Four Paradigms}
\bc
\begin{tabular}{r l}
Imperative & Object-Oriented \\
Logic & Functional
\end{tabular}
\ec
\begin{annotation}
Stuff.
\end{annotation}
\end{frame}

\begin{frame}
\frametitle{History}
\bi
\item History
\ei
\begin{annotation}
Stuff.
\end{annotation}
\end{frame}

\begin{frame}
\frametitle{Summary}
\bi
\item summary
\ei
\medskip
{\scriptsize Produced using the {\LaTeX} Beamer package, along
  with other free-software programs.}
\end{frame}

\end{document}
